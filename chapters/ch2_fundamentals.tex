\chapter{Fundamentals}
This chapter presents the fundamental concepts 


\section{Mathematical Optimization}

\iffalse
However, before introducing the pattern recognition method, Hart describes the NILM problem as a combinatorial optimization (CO) problem, formulated in the following way: let's assume that the measured value (current/power) in the input of the house is $P(t)$ at time $t$. The goal of the algorithm would be to decode $P(t)$ in many components $P_i(t) \ \forall \ i \in 1..n$, where $n$ is the number of power components and each one would be be associated to one specific state $i$. Hence, we have that

$$P(t) = P_1(t) + ... + P_n(t)$$
 
The CO formulation can be written by using the equation \ref{co} for each time reading $t$ 

\begin{equation} \label{co}
    \min_{x} \quad \left|P(t) - \sum_{i=1}^{n} x_i\ P_i \right|
\end{equation}

where $x_i(t)$ is a boolean that describes the state of the power component $i$ at time $t$. As noted in the reference, this problem although mathematically attractive, is a NP-complete “weighted set” problem, hence it has a high computational cost. In addition, it's complexity increases with the addition of more states of devices or measurements.
\fi


Mathematical optimization (also known as mathematical programming) is the process of minimization or maximization of an objective function of many variables, subject to constraints on the variables \cite{ampl}. The equation \eqref{opt_example} shows an example of optimization problem.

\begin{equation} \label{opt_example}
    \max_{x_i} \quad \sum_{i\ = 1}^{n} v_i\ x_i
\end{equation}

subject to 

\begin{equation}
    \sum_{i\ = 1}^{n} w_i\ x_i \ \leq \ W \quad \forall \ x \in \{0,1\}
\end{equation}

Equation \eqref{opt_example} is a classic \textit{integer programming} problem also known as the knapsack problem. Given a set of $i = 1..n$ different items, where each item has a specific value $v_i$, the main goal is to maximize the overall profit given that each item has a weight $w_i$ and it is not possible to overcome the maximum limit $W$. The variable of this problem $x_i$ is an integer binary variable, hence, the name of this type of problem. Some of the major subfields for mathematical optimization are \cite{cplex, ampl}:
\begin{itemize}
    \item \textbf{Linear Programming:} In this type of problem, the variables in the optimization function and in the constraints are linear. In the constrains, those variables are always assumed to be a linear combination of other variables. In the optimization funciton, the variables are always first order equations.  
    \item \textbf{Nonlinear Programming:} The objective function and/or the constraints contain nonlinear variables. A nonlinear variable could be a convex problem, for example, minimize the square of an error between. The product of two variables is also considered a non linear formulation. 
    \item \textbf{Integer Programming:} Linear problem in which all the variables are assumed to take an integer value. This value could be either a binary value or an natural number. 
    \item \textbf{Quadratic Programming:} Allow quadratic terms in the objective function. Most times, this is a convex problem.
    \item \textbf{Mixed-Integer Linear Programming:} Integer programming problem which contains a linear objective function (without quadratic terms).
\end{itemize}

\cite{cplex}. 

\subsection{Optimization Software}
ampl
gams
knitro
cplex

\subsection{Linearization of an Absolute Objective Function}
Boyle in cite{boyle} demonstrates how to linearize the \textit{Chebyshev approximation problem} defined as:

\begin{equation} \label{chebyshev}
    \min_{x} \max_{i=1,...,k} \quad \left| a_i^T\ x - b_i \right |
\end{equation}

\section{NILM as a Combinatorial Optimization Problem}

The energy disaggregation problem can be formulated assuming that the measured variable (current or power) in the input of the house is given by $P(t)$, for each time $t$. As shown in \eqref{Ps}, the objective of the NILM would be to decode $P(t)$ in power states $P_i(t)$, $\forall \ i \in \{1,...,n\}$. Where $n$ is the total number of power states for all appliances. 

\begin{equation} \label{Ps}
    P(t) = P_1(t) + ... + P_n(t)
\end{equation}
 
Each appliance is associated with one or more power states. For example, an ON/OFF appliance (e.g., a toaster) could be represented by a single state, while a washing machine could be represented by multiple states, since its power consumption changes over time. 
Eventually, the classical NILM problem can be rewritten as the optimization problem shown in the equation \eqref{classic}. Where $x_i(t) \in \left\{ 0 , 1 \right\}$ is a boolean variable that decides the status of the power state $i$, at time $t$ \cite{hart}.

\begin{equation} \label{classic}
    \min_{x} \quad \left| P(t) - \sum_{i=1}^{n} x_i(t)\ P_i \right |
\end{equation}

Equation \eqref{classic} aims at finding the combination of power states $P_i$ that best approximate the measure $P(t)$. When other types of measurements are also available (such as reactive power, harmonics or distortion factor), the classical problem in \eqref{classic} may include these measurements in a vector. Equation \eqref{classic2} also includes the reactive power measurement $Q(t)$ and reactive power states $Q_i$. 

\begin{equation} \label{classic2}
    \min_{x} \quad \left|\ \begin{bmatrix}
         P(t) \\
         Q(t) \\
        \end{bmatrix} - \sum_{i=1}^{n} x_i(t)\ \begin{bmatrix}
         P_i \\
         Q_i \\
        \end{bmatrix} \ \right|
\end{equation}

While the author x mentions that this approach is not viable due to the explosion of features, i.e., high computational time, the author considered a scenario in which the input table contains over x inputs. As described later, this work does not requires a big number of features and it runs in a fair computational time. In addition, there are constraints for atenuating the computational influence. 

\iffalse
Responder à estas críticas:


- Optimisation is computationally intractable
George Hart, one of the early pioneers of disaggregation research, points out that the optimisation
problem specified in equation 7.2 is an NP-complete “weighted set” problem and that
a precise solution is only achievable by enumerating every possible state (G. W. Hart 1992).
This is computationally impractical because n appliances, each of which can occupy any one
of s states, can be configured in s
n
combinations so the computational complexity blows up
exponentially as O(s
n
). Say we have thirty appliances, each of which can be in one of four
states, and we have a month of data sampled once every five seconds. That is approximately
1024 operations1
, which would take 5×1010 seconds (∼ 1 700 years) on NVIDIA’s top-of-the-line
GPU at the time of writing2
.
Whilst the optimisation problem specified in equation 7.2 is a succinct description of the problem,
it fails to capture many of the challenges present in practical systems. These problems
include (but are not limited to):
1. We are unlikely to know the power consumption of every appliance.
2. We are unlikely to know the total number of appliances.
3. Many appliances do not draw tidy, discrete levels of power; instead their power consumption
may spike, undershoot, oscillate or ramp over time.
4. A smart meter may sample less frequently than is required to faithfully capture rapid
changes. In other words, the meter may sample at sub-Nyquist rates. This results in
considerable distortion of the digital recording.
5. Many appliances (like washing machines and tumble driers) have multiple internal states.
Transitions between these states may be non-deterministic. Each run of the appliance
may produce a different waveform (see Figure 7.1).
6. Different appliances of the same class produce different waveforms (which is a problem if
we want to build a common database of appliances for multiple users).
7. Some appliances generate identical waveforms (e.g. a kettle and the water heater in a
washing machine generate very similar waveforms).
8. Appliance signatures overlap and occlude each other in the aggregate smart meter signal.
9. The mains voltage in the UK is nominally 230 volts but can range from 216 volts to
253 volts which is -6%, +10% of the nominal 230 volt supply voltage3
. Assuming a linear
load, we can expect the power consumption to vary by -12%, +20%. Home energy meters
do not measure voltage, but utility-installed smart meters do.
10. Ultimately users care more about how much energy each appliance uses rather than when
each appliance is on. Estimating energy consumption for a simple two-state appliance
like a toaster is trivial if we know how long the appliance has run for. But estimating
power consumption of complex appliances like washing machines is less trivial.
These challenges mean that conventional optimisation approaches such as combinatorial optimisation
are not feasible for anything other than toy scenarios.


\fi

\section{NILM as a Pattern Recognition Problem}
One of the most popular approaches for solving the NILM problem is pattern recognition. The technique is fundamentally based on finding pairs of edges in the signal with opposite directions. The figure x describes the first algorithm, proposed by Hart in x. 

\section{Other Approaches}
- HMM


Most of the successive works were focused on alternative edge detection strategies and increasing the number of features to be used besides the active and reactive power from the original work. Those alternative features includes harmonics, time of operation, hour of activation, etc. There are also features extracted from high frequency data such as the senoidal signal. However, it is worth noting that many of those features are not available in smart meters. Hence, NILM strategies should also be suitable to cases with limited data. 

\section{Summary}