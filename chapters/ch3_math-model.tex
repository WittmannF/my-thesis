
\chapter{Math Load Model}

[MOTIVACAO]
- Very few progress was observed in the CO formulation of the NILM problem presented in the previous chapter. 
- The previous chapter described the major formulations of the NILM problem. In regards to the optimization formulation, very low attention has been paid by researchers due to the high computational cost. However, thanks to the advance in computation, an expansion of the CO problem is possible for a fair number of states. 

[INTRO]
- This chapter presents a set of constraints and strategies for modeling the signature of a load
- This chapter presents a MILP formulation for the NILM problem. 
- This chapter presents a set of strategies and constraints for improving the performance of the CO problem. 
- This chapter presents a new set of constraints and strategies for expanding the classic CO problem. 


Those are the main contribution of this dissertation. First the classic NILM optimization problem is reformulated as a MILP problem. Next, three constraints are proposed for defining modeling a load signature such as the transition of states and minimum time. Strategies and constraints for decreasing the computation time are also introduced. Finally, the fulll proposed model is presented. 

\section{Mixed-Integer Linear Programming (MILP) Formulation}

The Equation \eqref{classic2} can be reformulated as a MILP problem, shown in \eqref{milp2}--\eqref{eq7}. This way, the new formulation becomes suitable for linear solvers.

\begin{equation} \label{milp2}
    \min_{x_i(t)} \quad \sum_{t\ \in\ T} \delta_P(t) + \delta_Q(t)
\end{equation}

subject to 

\begin{equation} \label{eq65}
    P(t) - \sum_{i\ = 1}^{n} x_i(t)\ P_i \ \leq \ \delta_P(t) 
\end{equation}

\begin{equation} 
  P(t) - \sum_{i = 1}^{n} x_i(t)\ P_i \ \geq \ -\delta_P(t)
\end{equation}

\begin{equation} 
   Q(t) - \sum_{i\ = 1}^{n} x_i(t)\ Q_i \ \leq \ \delta_Q(t)
\end{equation}

\begin{equation} \label{eq7}
  Q(t) - \sum_{i = 1}^{n} x_i(t)\ Q_i \ \geq \ -\delta_Q(t)
\end{equation}

Where $\delta_P(t)$ and $\delta_Q(t)$ in \eqref{milp2} are the difference between the aggregated power states and the actual measurement (i.e., the approximation error). Besides the linear formulation, another difference from \eqref{classic2} is that the time $T$ is included in the objective function. The advantage of doing this is the possibility of adding new time-dependent constraints in order to improve the model's accuracy. 

\subsection{Window-based formulation}
The time set $T$ increases the computational burden since the variable $x_i(t)$ depends on the number of time periods. To overcome this issue, it is possible to separate the set of time periods into smaller, homogeneous windows. Let $\tilde{T} \subset T$ be the set of time periods within a window, where $\tilde{T} = \left\{ T_{0} , T_{0} + 1, \cdots , T_{0} + m \right\}$, $m$ is the number of time steps of the window, and $T_{0}$ is the initial period for each window. The MILP problem can now be written using a sequenced  optimization process as shown in Algorithm \ref{algorithm1}, where $T_{\text{end}}$ is the final period of $T$.

\begin{algorithm}\label{algorithm1}
\SetAlgoLined
 \Repeat{$T_{0}$ > $T_{\text{end}}$}{

let  $\tilde{T} = \left\{T_{0}, \cdots, T_{0}+m \right\}$

solve
\begin{equation}
    \min_{x_i(t)} \quad \sum_{t\ \in\ \tilde{T}} \delta_P(t) + \delta_Q(t)
\end{equation}

s.t.:

\begin{equation} \label{first_eq}
    P(t) - \sum_{i = 1}^{n} x_i(t)\ P_i \ \leq \ \delta_P(t)
\end{equation}

\begin{equation}
  P(t) - \sum_{i = 1}^{n} x_i(t)\ P_i \ \geq \ -\delta_P(t)
\end{equation}

\begin{equation} \label{third_eq}
    Q(t) - \sum_{i = 1}^{n} x_i(t)\ Q_i \ \leq \ \delta_Q(t)
\end{equation}

\begin{equation} \label{fourth_eq}
  Q(t) - \sum_{i = 1}^{n} x_i(t)\ Q_i \ \geq \ -\delta_Q(t)
\end{equation}


let $T_0 = T_0 +m +1$
 }
 \caption{NILM using a window-based algorithm}
\end{algorithm}

Algorithm 1 is still equivalent to the classic optimization formulation and will be used as one of the approaches to be compared in the results. From now on, Algorithm 1 will be referred as the CO method. 

One more constraint that is helpful for cases in which the reading in the window is less than a given threshold (for example $TH = 30W$). The constraint \eqref{set_zero} allows to automatically set to zero the binary variables under those circumstances. 

\begin{equation} \label{set_zero}
   x_i(t) = 0 \quad \forall t \in \tilde{T}, i \in S | P(t) \leq TH
\end{equation}

\section{Load Signature Constraints}

In this section, a new set of constraints are presented to model the load signatures and to communicate the optimization process of one window with the others. First, let's consider a practical example. Fig.~\ref{example} shows the load signature of two hypothetical appliances: a washing machine and a stove.

\begin{figure}[tb]
    \centering
    \includegraphics[width=1\columnwidth ]{example_2.pdf}
    \caption{Two hypothetical load signatures to be modeled}
    \label{example}
\end{figure}

The washing machine's load signature can be approximated using three power states ($P_1(t)$, $P_2(t)$ and $P_3(t)$) while the stove's has one single power state ($P_4(t)$). In order to model these loads, the following hypotheses are considered: 

\begin{itemize}

\item Multiple states $P_i(t)$ (also multiple reactive power states, if available) from the same appliance cannot be activated simultaneously;
\item Some loads work as finite-state machines, i.e., a given state is only activated if another state of the same appliance has finished;
\item A state $i$ can have a minimum time $MD_i$ in which it should remain ON.

\end{itemize}

The former hypotheses are used to formulate a set of constraints used to efficiently represent the load signatures within the proposed MILP model. A detailed analysis of each one of the former hypotheses is shown in the following subsections.

\subsection{Avoiding Multiple States From the Same Appliance}

Parameter $D_i$ identifies the appliance's index of each state $i$. For example, the states $i = 1, 2, 3$ in Fig.~\ref{example} are associated to the washing machine, that is, $D_1 = D_2 = D_3 = 1$, where the number 1 means ``washing machine''. Likewise, $D_4$ refers to the stove, identified by the number 2. In order to avoid simultaneously allocation of different power states from the same appliance, constraint \eqref{overl} can be included.

\begin{equation} \label{overl}
   \sum_{i \in S | D_{i} = j} x_i(t) \leq 1 \quad \forall t \in \tilde{T}, j \in D
\end{equation}

Constraint \eqref{overl} limits the sum of the states $x_i(t)$ from the same appliance to one. This way, $x_1(t)$, $x_2(t)$ or $x_3(t)$ cannot be simultaneously activated since $D_1$, $D_2$ and $D_3$ are associated to the same appliance.

\subsection{Linking the Transition Between Power States}

In Fig.~\ref{example}, power state $P_2(t)$ should be ON only if the power state $P_1(t)$ has finished. Likewise, we could also fix the power state $P_3(t)$ to be ON only if $P_2(t)$ has finished. The goal is to include a constraint that allows a specific power state to be activated only if a previous one (from the same appliance) has finished. To do so, two binary variables are used to determine the transition from a ON state to an OFF state, and vice versa. The two variables are called $up_i(t)$ (turned ON) and $dw_i(t)$ (turned OFF). Then, the linking constraints are given by \eqref{updw}--\eqref{updw2}.
%
%\vspace{-10pt}

\begin{equation} \label{updw}
    x_i(t) - x_i(t-1) = up_i(t) - dw_i(t) \quad \forall i \in S, t > T_0
\end{equation}


\begin{equation} \label{updw3}
    x_i(T_0) - X_i = up_i(t) - dw_i(t) \quad \forall i \in S, t = T_0
\end{equation}

\begin{equation} \label{updw2}
    up_i(t) + dw_i(t) \leq 1 \quad \forall i \in S, t \in \tilde{T}
\end{equation}

\noindent where $X_i$ saves the state $x_i(t)$ of the last time period of each window, necessary for the initialization of $x_i(t)$ in the next window. In constraint \eqref{updw}, $up_i(t)$ will be 1 only if the decision variable $x_i(t)$ makes a transition from 0 to 1, at time $t$. Likewise, $dw_i(t)$ will be 1 only if $x_i(t)$ makes a transition from 1 to 0, at time $t$. Next, \eqref{updw3} links the state transitions between windows. Finally, \eqref{updw2} prevents $up_i(t)$ and $dw_i(t)$ to be simultaneously 1.

%Now that we have these two variables, let's consider that we save the transition of states in a parameter called $\text{prev}_i$. In the Fig. \ref{example} we have for example in the washing machine that the value of $prev_2$ is 1 since we want to fix the second state to happen only after the first one. We can fix state transitions using the following constraint:

Using constraints \eqref{updw}--\eqref{updw2} and the parameter $\text{prev}_i$, we can now link the transition between two states using the additional equation in \eqref{state}.

\begin{equation} \label{state}
    up_i(t) = dw_{\text{prev}_i}(t) \quad \forall i \in S, t \in \tilde{T} \ | \ \text{prev}_i>0
\end{equation}

As an example, the state $i = 2$ of the washing machine in Fig.~\ref{example} can only change from OFF to ON (i.e., $up_2(t) = 1$) if the state $i = 1$ has change from ON to OFF (i.e., $dw_1 = 1$), at a given time $t$.

\subsection{Minimum Active Time}

One last hypothesis that is proposed in this paper is setting a minimum active time of a state. Parameter $MD_i$ in Fig.~\ref{example} establishes the minimum number of time samples in which the state $i$ should be kept activated. The set of constraints presented in \eqref{md1}--\eqref{md3} are proposed to carry out this process. 

\begin{equation}\label{md1}
    \sum_{k\ =\ T_0}^{G_i} \left[ 1 - x_i(k) \right] = 0 \quad \forall i \in S
\end{equation}

\begin{multline}\label{md2}
    \sum_{k\ =\ t}^{t+MD_i-1} x_i(k) \geq MD_i \left[ x_i(t) - x_i(t-1) \right] \\
    \forall i \in S, t \in G_i + T_0 \hdots T_f - {MD}_i + 1
\end{multline}

\begin{multline} \label{md3}
    \sum_{k\ =\ t}^{T_f} \left\{ x_i(k) - \left[ x_i(t) - x_i(t-1) \right] \right\} \geq 0 \\
    \forall i \in S, t \in T_f - MD_i + 2 \hdots T_f 
\end{multline}

Constraint \eqref{md1} activates a certain state that was already activated at the end of the previous window, but for less than $MD_i$ samples. Parameter $G_i$ is the number of periods in which the state $i$ must remain ON at the beginning of the window. It is calculated as $G_i = \min \left\{ T_f, \left[ MD_i - NP_i \right] X_i \right\}$. $NP_i$ is the number of time periods in which $i$ has been activated in the previous window, given by the equation \eqref{np}

\begin{equation} \label{np}
    NP_i = \sum_{k\ =\ T_f - MD_i + 2}^{T_f} { x_i(k) } \quad \forall i \in S
\end{equation}

Constraint \eqref{md2} forces $x_i(t)$ to be 1 for at least $MD_i$ time samples. Finally, constraint \eqref{md3} is used to represent the operation at the final portion of the window, when there are less than $MD_i$ samples available. It forces a given state $x_i(t)$ to be ON until the end of the window, only if it has been activated at any moment within this final interval. The size of $MD_i$ has as limit the length $m$ of the window, i.e, $MD_i \leq m$. 
As a side note, the set of equations presented in \eqref{overl}--\eqref{md3} are similar to models of the operation of thermal units in the unit commitment problem, proposed by authors in \cite{carrion2006}.

\subsection{Full Proposed Model}

The full proposed model is given by the Algorithm \ref{algorithm2}. As it will be illustrated in the next section, this kind of optimization problem can be solved with the help of standard convex mathematical optimization software. 

\begin{algorithm}[H]\label{algorithm2}
\SetAlgoLined
 \Repeat{$T_0$ > $T_f$}{

let  $\tilde{T} = \left\{T_{0}, \cdots, T_{0}+m \right\}$

solve
$$\min_{x_i(t)} \quad \sum_{t\ \in\ \tilde{T}} \delta_P(t) + \delta_Q(t)$$

s. t. 
\begin{center}
$(\ref{first_eq})$ ... $(\ref{md3})$
\end{center}

let $T_0 = T_0 +m$ + 1
 }
\caption{Proposed NILM using a window-based algorithm.}
\end{algorithm}

\vfill


\section{Summary}
- This chapter has presented a set of constraints for modeling the load signature, which are the main contributions of this work. 
- This chapter has expanded the classic NILM CO model for modeling load signatures in a computationally efficient way. 
- First, the classic CO problem was reformulated to a MILP problem. Next, a window based formulation was presented in order to decrease the computational burden. Then, constraints were introduced for modeling the load signature based on three features: power state, minimum time and sequence of states. As we will see in the next chapter, those features can be extracted from a model in both a supervised and unsupervised setting. 