    
\newpage

\begin{center}
  \large{\textbf{RESUMO}}
\end{center}

% FIXME Remover deste ponto até a linha 14.
Este trabalho apresenta um método de monitoramento não intrusivo (Non-Intrusive Load Monitoring - NILM) baseado em programação linear inteira mista. NILM são métodos para decompor leituras de medidores de energia em informações a respeito dos aparelhos eletrodomésticos em operação. Tais informações, como consumo e estado de operação, são valiosas para promover a eficiência energética e manutenção preventiva. A técnica NILM proposta neste trabalho expande o modelo clássico fundamentado em otimização combinatória (Combinatorial Optimization - CO). A nova formulação lida com o problema de ambiguidade de cargas similares, presente no modelo clássico. Restrições lineares são utilizadas para representar eficientemente as assinaturas de carga. Além disso, uma estratégia baseada em janelas temporais é proposta para melhorar o desempenho computacional. A desagregação de cargas pode ser feita utilizando apenas medidas de potência ativa em uma baixa taxa de amostragem, disponível na maioria dos medidores de energia. A técnica é também adequada para uma abordagem sem supervisão. O desempenho do algoritmo é validado utilizando três casos de teste a partir da base de dados pública AMPds. Para a abordagem sem supervisão é utilizada uma base de dados residencial privada. A taxa de amostragem do caso de teste é de uma amostra por minuto. Os resultados demonstram a habilidade do método proposto para identificar e desagregar com precisão as assinaturas de energia individuais de forma computacionalmente eficiente. 

\vspace{.2cm}
\textbf{Palavras-chave}:
% FIXME Remover a linha abaixo.
desagregação de carga, assinatura de carga, programação linear inteira mista, monitoramento não intrusivo, otimização.
% TODO Inserir as palavras-chave aqui.

\vspace{3cm}
\newpage

\begin{center}
  \large{\textbf{ABSTRACT}}
\end{center}

This work presents a non-intrusive load monitoring (NILM) method based on mixed-integer linear programming (MILP). NILM are methods for decomposing measurements from energy meters into information regarding appliances in operation. Such information, such as the the power consumption and operating state, are valuable for promoting energy savings and predictive maintenance. The proposed technique expands the classical NILM combinatorial optimization (CO) model. The new formulation handles the problem of ambiguity of similar loads, present in the classic model. Linear constraints are used to efficiently represent load signatures. Additionally, a window-based strategy is proposed to enhance the computational performance of the proposed NILM algorithm. The disaggregation can be made using only active power measurements at low sampling rate, which is available at most energy meters. In addition, to improve the method's accuracy, other features can be added to the model, if available, such as reactive power or harmonic measurements. The technique is suitable to an unsupervised NILM approach. The performance of the algorithm is evaluated using three test cases from the public dataset AMPds. The unsupervised approach uses a private residential dataset. The sampling rate from the test case is 1 sample per minute. Results demonstrate the ability of the proposed method to accurately identify and disaggregate individual energy signatures in a computationally efficient way.


\vspace{.2cm}
\textbf{Keywords}:
% FIXME Remover a linha abaixo.
load disaggregation, load signature, mixed-integer linear programming, non-intrusive load monitoring, optimization.